Como arquitectura de desarrollo se escogió MVVM (Modeo Vista Vista Modelo) por
medio del framework Volt.
Esta nos permitirá crear una aplicación mas rica e interactiva para
el usuario final, sobre todo en la parte de edición de música, en la cual
es necesaria el almacenamiento/actualización de información de forma
dinámica (sin refrescos de pagina).
Volt también divide a un proyecto web en múltiples aplicaciones, esto con
el objetivo de reutilizar código como librerías, en nuestro caso se ha
planteado crear una aplicación separada para la edición de loops, esta
contendrá la funcionalidad necesaria para crear un editor y los controles para
crear secuencias y grids para la programación de estas secuencias. El código
HTML para generar secuencias y grids puede ser reutilizado de cualquier manera
dentro de otra aplicación Volt, en nuestro caso podremos crear una pagina de loops
personalizada con el numero de secuencias y grids que deseamos, reutilizando los
componentes creados dentro de la aplicación compartida.
Como motor de todo el sistema de reproducción de sonidos se creo una librería
(Opal Music), la cual esta escrita en el lenguaje ruby, pero que por debajo
compila a javascript. Esta se encarga de manipular la librería Web Audio
Api para la creación de notas, secuencias y programación de secuencias.
A continuación se describe los componentes indicados anteriormente:

\section{Aplicación Music}

La aplicación 'music' almacena toda la lógica necesaria para crear
un editor básico, en nuestro caso un loop. Consta de los siguientes elementos:

\subsection{Modelos}
\subsubsection{Loops (canción)}
El loop es el modelo principal dentro del editor, se lo puede considerar como
una canción pero para efectos de la aplicación se le considera un elemento de
reproducción o tono que puede ser reutilizado, consta de los siguientes elementos:
\paragraph{Title (titulo)}
Corresponde al titulo que le vamos a dar a nuestro loop
\paragraph{Tempo (velocidad)}
Es el valor dado para el aumento o disminución en la velocidad de reproducción
(numero de notas por segundo)
\paragraph{Volume (volumen)}
El volumen permitirá almacenar este valor el cual sera usado para todo las
secuencias dentro del loop
\paragraph{Size (tamaño)}
El loop tendrá en un tamaño especifico, el cual determinara el numero de repeticiones
que pueden tener cualquier secuencia a lo largo del loop, su valor mínimo
es de cuatro repeticiones

\subsubsection{Sequences (secuencias)}
La secuencia forma parte de un loop y es el elemento principal para generar sonidos
dentro de la aplicación, consta de los siguientes elementos:
\paragraph{Title (titulo)}
Nos sirve para identificar cada una de las secuencias dentro de un loop, también
es usado dentro del grid, para identificar las secuencias en cada fila del mismo.
\paragraph{Volumen (volumen)}
Representa el volumen de reproducción de la secuencia, esta también puede ser tomada
del loop al que pertenece, pero se decidió dejarlo por cuestión de usabilidad de
una secuencia (fuera del contexto de un loop).
\paragraph{Notes Number (numero)}
Es la cantidad de notas que tendrá nuestra secuencia, sus valores pueden varias entre los
siguientes valores: 1, 2, 4 u 8
square y sawtooth
\paragraph{Effect (efecto)}
Es el tipo de efecto con el cual se reproducirá la secuencia, para efectos prácticos
este valor almacenara los tipos de ondas proporcionados por Web Audio API: sine,
square y sawtooth

\subsubsection{Cells (células)}
Las células forman parte de una secuencia, sirven para marcar su reproducción o no
a lo largo de un loop, consta de un solo valor: 'value' el cual es un valor de
verdadero o falso. La cantidad de los mismos puede aumentar o disminuir en una
secuencia tomando el valor del tamaño del loop. Se decidió crear este modelo por
fuera del campo secuencia para poder detectar sus cambios dinámicamente, en Volt
los únicos cambios que pueden ser detectados reactivamente son los valores de modelos

\subsubsection{Notes (notas)}
Las notas también forman parte de una secuencia y al igual que las células
los cambios de sus valores deben ser realizados dinámicamente, por lo cual
cuentan son su propio modelo. La longitud de este valor cambia de acuerdo
al numero de notas almacenado en la secuencia

\subsection{Controladores (Vista Modelos)}
\paragraph{Controlador Loops}
Se encarga de almacenar y actualizar dinámicamente la cantidad de secuencias creadas,
así como controlar los valores de volumen, reproducción y tempo.
Cuando se hacen cambios en el loop estos se ven reflejados en el momento de la reproducción
(calculados por medio de los valores del loop y de cada una de las secuencias). A su
actualiza los cambios realizados en la grilla de reproducción, por medio de las células
de cada secuencia.

\paragraph{Controlador Secuencias}
Al igual que el controlador de loops este se encarga de actualizar dinámicamente los
valores de cada secuencia. A su ves realiza las tareas mas importantes dentro del editor:
ejecuta la reproducción de notas con sus respectivas configuraciones(volumen, efecto, tempo,
momento de ejecución) por medio de la librería Opal-Audio.

\subsection{Vistas}
\subsubsection{Editor}
Los datos contenidos dentro de esta vista son creados y actualizados por medio del
controlador. Este mostrara los datos loop que se quiera visualizar, y a su ves se
actualizara reactivamente cuando se realizan cambios a los datos (sin necesidad de
refrescar la pagina).
\subsubsection{Grid}
Este a su ves es manejado por el controlador de loops, para poder ser desplegado
necesita usar los datos del loop ingresado. Esta vista esta contenida dentro
la vista Editor
\subsubsection{Controls (controles de secuencia)}
Esta vista, la cual es manejada por el controlador de secuencias, despliega todo los
controles de una secuencia especifica, los cuales son necesarios para cambiar
sus valores al momento de reproducir las notas. Hace uso de los
siguiente controles: volumen, efecto, reproducción(play) y cambio de notas. Los
controles de una secuencia pueden ser usados independientemente o embeberlos
dentro de la vista del editor

\section{Aplicación Main(principal)}

\subsection{Modelos}
\subsubsection{Users (usuarios)}
El usuario es el modelo pilar dentro del sistema, se encargara de almacenar tanto
las relaciones de amistad como los datos del mismo. Este modelo a su
ves contiene los loops creados.
\paragraph{Name (nombre)}
Almacena el nombre del usuario en una cadena de caracteres
\paragraph{Password (contraseña)}
Almacena en una cadena de caracteres encriptado la contraseña necesaria para
el inicio de sesión.

\subsection{Controladores}
\subsubsection{Controlador de Notificaciones}
Cada mensaje de notificación sera creado por este controlador cuando se haya creado
o borrado un dato en el usuario (estos cambios pueden ser de amistades o loops),
la creación de estas notificaciones se lo hace modificando internamente el comportamiento en las
actualizaciones de estos datos por medio de Volt.

\subsubsection{Controlador de Loops de Usuario}
Este se encarga se suministrar los datos necesarios para mostrar los
loops pertenecientes a un usuario, también permite crear o eliminar los
mismos de manera reactiva (actualizara la lista de loops sin refrescar
la pagina)

\subsection{Vistas}
\subsubsection{Notificaciones}
Mostrara los mensajes de notificación junto con la fecha y el usuario al
que refiere la actualización
\subsubsection{Lista de loops}
Tendrá los controles necesarios tanto para crear un nuevo loop o
eliminar uno existente, este ultimo dentro de la lista de loops.

\section{Libreria Opal-Music}
Esta librería fue creada como una gema(librería) del lenguaje Ruby. Por
debajo compila al lenguaje Javascript por lo cual solo puede ser usada
dentro del navegador(por medio de la librería Opal), en nuestro caso
para la creación de tonos musicales por Web Audio Api. El motivo para
crear esta librería en Ruby es por la usabilidad dentro del framework
Volt, el cual también admite juntar código Javascript dentro de los
controladores, pero también nos brinda la posibilidad de usar ruby puro
dentro del navegador. Esta librería consta de los siguientes componentes:

\subsection{Note (nota)}
Se encarga de inicializar una nota con una determinada duración,
por debajo contiene una lista con las siglas de las notas y sus respectivas
frecuencias.
\subsection{Sequence (secuencia)}
Se encarga de la reproducción de múltiples notas con un determinado
volumen. Recibe como parámetros un 'AudioContext', el cual sirve como
contexto para la creación de elementos sonoros de Web Audio Api. Por
cada nota este componente crea un 'Oscillator' perteneciente al
contexto de audio ingresado.
\subsection{Schedule (Programación de Secuencias)}
Programa una misma secuencia y las repite a lo largo del tiempo.
Por debajo utiliza una cadena de valores Verdadero/Falso para programarlo.
