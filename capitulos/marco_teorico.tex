\section{Metodologías Ágiles}

En el software se tiene un amplio número de metodologías de desarrollo, las
primeras fueron inspiradas en metodologías usadas en otras ramas de ingeniería
y adaptadas para su uso en la construcción de sistemas informáticos. Estas
promueven un desarrollo lineal, haciendo énfasis en la planificación exhaustiva
para su posterior ejecución. Un ejemplo de estas metodologías es el desarrollo
en cascada.

Uno de los motivos por los cuales no se recomienda usar metodologías de
desarrollo lineales, como el desarrollo en cascada, es que son procesos muy
rígidos,  los cuales  contemplan, de antemano, que todos los proyectos de
software son similares y que no necesitan gestionar muchos cambios a lo largo
de su existencia.

Son contados los tipos de proyectos en el mundo del software que tienen una
base y aplicación parecidas, y al ser el desarrollo de sistemas una rama
relativamente joven en comparación a otras ingenierías, no se puede confiar
ciegamente en un proceso que considera la construcción de sistemas de software
como un proceso probado y casi infalible. Las metodologías ágiles nacieron con
el objetivo de solventar estas deficiencias y obtener un mejor tiempo de
respuesta a posibles errores o cambios que se den a lo largo de un proyecto,
por medio de iteraciones cortas para así obtener un mejor feedback por parte
del cliente.

\subsection{Manifiesto Ágil}

\subsubsection{Tipos}

\paragraph{Scrum}

\section{Desarrollo guiado por pruebas}

En el software se tiene un amplio número de metodologías de desarrollo, las
primeras fueron inspiradas en metodologías usadas en otras ramas de ingeniería
y adaptadas para su uso en la construcción de sistemas informáticos. Estas
promueven un desarrollo lineal, haciendo énfasis en la planificación exhaustiva
para su posterior ejecución. Un ejemplo de estas metodologías es el desarrollo
en cascada.

Uno de los motivos por los cuales no se recomienda usar metodologías de
desarrollo lineales, como el desarrollo en cascada, es que son procesos muy
rígidos,  los cuales  contemplan, de antemano, que todos los proyectos de
software son similares y que no necesitan gestionar muchos cambios a lo largo
de su existencia.

Son contados los tipos de proyectos en el mundo del software que tienen una
base y aplicación parecidas, y al ser el desarrollo de sistemas una rama
relativamente joven en comparación a otras ingenierías, no se puede confiar
ciegamente en un proceso que considera la construcción de sistemas de software
como un proceso probado y casi infalible. Las metodologías ágiles nacieron con
el objetivo de solventar estas deficiencias y obtener un mejor tiempo de
respuesta a posibles errores o cambios que se den a lo largo de un proyecto,
por medio de iteraciones cortas para así obtener un mejor feedback por parte
del cliente.

\subsection{Red, Green, Refactor}

El desarrollo guiado por pruebas considera un proceso estricto, el cual se debe
seguir, al pie de la letra, con el objetivo de pensar primero lo que queremos
implementar, para así tener una mejor idea de lo queremos desarrollar, y luego
ir mejorando nuestro código, a este proceso se lo denomina Red, Green, Refactor

\subsubsection{Red (prueba fallida)}

El Red se refiere a que debemos crear primero nuestra prueba y no realizar
ningún código. Al correr el test nuestra prueba debería estar en rojo, lo cual
significa que el test no ha pasado.

\subsubsection{Green (prueba superada)}

El siguiente paso es hacer pasar nuestra prueba creando el código
correspondiente, una vez que el test pase, al correr nuestra prueba esta dará
verde.

\subsubsection{Refactor (mejorar código)}

Una vez que hemos acabado de programar procedemos a mejorar nuestro código,
cada vez que realizamos un refactor debemos aseguramos de que nuestra prueba
siga pasando. Si nuestro test vuelve a dar error repetimos el ciclo.

\subsection{Errores comunes en el proceso de crear pruebas}

\subsubsection{Pensar en el porcentaje de cobertura de nuestras pruebas}
La cobertura de pruebas es el porcentaje de código que cubren nuestros test. La
situación ideal es tener una cobertura del 100%, pero esto nunca se logra ,
tanto por los cambios constantes que se realiza al código y por el hecho mismo
de que las personas escriban los tests. Las pruebas que se enfocan solo en la
cobertura tienden a crear código poco refactorizable y generan pruebas muy
frágiles y poco tolerantes a los cambios

\subsubsection{No fallar primero}
La cobertura de pruebas es el porcentaje de código que cubren nuestros test. La
situación ideal es tener una cobertura del 100%, pero esto nunca se logra ,
tanto por los cambios constantes que se realiza al código y por el hecho mismo
de que las personas escriban los tests. Las pruebas que se enfocan solo en la
cobertura tienden a crear código poco refactorizable y generan pruebas muy
frágiles y poco tolerantes a los cambios

\subsubsection{No tomar un tiempo comparable para realizar refactor}

Idealmente se debe tomar el mismo tiempo tanto para tanto hacer refactoring
como para crear el test. Esto a la larga produce un código mas limpio y
mantenible

\subsubsection{El código es muy fácil o muy difícil de testear}

\subsection{Mocking}
Las pruebas deben crearse en base solo a la funcionalidad que vayan a
implementar , cualquier dependencia debe mantenerse al mínimo, tanto por
rendimiento del sistema como por diseño del código. Otro motivo para mantener
la dependencia al mínimo en tests es que no podemos probar una dependencia de
manera correcta , como por ejemplo llamadas a servicios web externos , estas
llamadas pueden cambiar constantemente y quebrar nuestros tests. Con el objetivo
de ocultar estas dependencias se crearon los mocks.  Un mock es un objeto
ficticio que reemplaza a un objeto real dentro de nuestro test

\subsubsection{Ventajas}
Nuestros tests corren mas rápido y no dependen uno de otro, lo cual permite
probar cada parte de manera aislada y con mayor facilidad

\subsubsection{Desventajas}
Crear muchos mocks puede llevar a la creación de pruebas no fiables, ya que
asumimos una salida en nuestro mock que puede cambiar en cualquier momento en
el objeto real

\subsection{Pruebas y su impacto en el desarrollo}

\subsubsection{Diseño del código}
El realizar las pruebas primero antes de escribir el código nos ayuda a
entender mejor nuestro código y su relación con otras partes de nuestro
sistema, esto genera una mejora drástica en el diseño. La forma en la que los
tests nos indican que pueden haber problemas de diseño es cuando el test se
vuelve muy difícil de implementar, lo cual nos indica que nuestro código tiene
muchas responsabilidades y parte de esas responsabilidades deben ser extraídas
y testeadas de manera mas profunda

\subsubsection{Documentación}
Las pruebas no solo nos ayuda a tener un código mas claro, lo cual en si ya
tiene un impacto en nuestro entendimiento del codigo, estas también sirven como
documentación, indicándonos cuales son los parámetros de entrada y cual es la
salida que obtendremos al correr un código especifico . Ciertas librerías de
pruebas nos permiten incluso escribir nuestros tests como si estuviéramos
escribiendo texto, dando lugar a pruebas claramente descritas.

\subsubsection{Debugging}
Uno de los objetivos de hacer pruebas es deshacernos
del proceso de debugging, si nos encontramos haciendo debugging significa que
no hemos realizado las suficientes pruebas a nuestro código y que este oculta
una responsabilidad que debe ser extraida, por lo cual se debe testear mas a
profundidad fuera del contexto de nuestro primer test

\section{Frameworks de desarrollo}
Un framework o marco de trabajo es una herramienta que permite crear
aplicaciones con mayor rapidez sin tener que empezar desde cero . Los marcos de
trabajo web incorporan conexiones a múltiples bases de datos, comunicación de
datos a la web, manejo de requests por parte de los navegadores y los presentan
de una forma sencilla para el programador. Esto lo logran siguiendo patrones de
diseño

\subsection{Tipos según patrón de diseño}
\subsubsection{Modelo Vista Controlador}
MVC es un patrón de diseño que separa su lógico por medio de tres objetos:
modelo , vista y controlador , el modelo se encarga de la parte de persistencia
de datos, la vista de la parte gráfica y el controlador de lo referente a
manejo de requerimientos activados por la vista.

\paragraph{Elementos MVC}

\subparagraph{Modelo}
El modelo abstrae todo lo referente a coneccion con bases
de datos por medio de un objeto, este objeto representa una tabla en nuestro
sistema. Cuando la clase modelo se hereda nos provee múltiples funcionalidades
como validaciones de datos, relaciones con otros objetos, y una interfaz clara
tanto como para guardar como para obtener datos de la base

\subparagraph{Vista}
En lo referente a web la vista representa el html creado. Algunos frameworks
incorporan templates que nos permiten generar html usando código desde nuestro
backend

\subparagraph{Controlador} Se encarga de manejar todos los requests
provenientes de la vista , en este caso un navegador será el que lo active. A
su vez redirecciona datos a las vistas correspondientes

\paragraph{Proceso MVC}
El controlador es el eje principal dentro de un esquema mvc. El ruteador es el
encargado de recibir un url y redirigirlo al controlador correspondiente. En
caso de que se necesiten datos para desplegar la pagina, el controlador recurre
al modelo correspondiente, el cual llamara a la base. Una vez que el
controlador obtiene los datos los redirige a la vista y los despliega en el
navegador

\paragraph{Ventajas}
Una de las principales bases de MVC y su mayor fortaleza es la separación
correcta de los roles y su flujo dentro de una aplicación, lo cual nos permite
pensar en el flujo de los datos a través de estas capas, sin tener que
preocuparnos por lo que pasa por debajo . Este concepto de mvc lleva ya
bastante tiempo , lo cual ha permitido la evolución de los frameworks que
adoptan este patrón a tal punto que ahora es muy fácil crear un aplicación
desde cero.

\paragraph{Desventajas}
MVC fue pensado para aplicaciones de escritorio , las cuales generalmente
manejan un solo lenguaje de programación. En el caso de un ambiente web se
tiene el lenguaje usado tanto en el backend como en el front-end, esto da a
lugar a aplicaciones mucho mas complejas que requieren mucha mas programación
fuera del framework, para crear aplicaciones mas dinámicas en el front-end.
Para crear aplicaciones que no necesitan refresco en el navegador, por ejemplo,
se necesita una librería externa , con su propio patrón de diseño, el cual a su
vez tiene que relacionarse con en el framework mvc para obtener datos.

\subsubsection{Modelo Vista Modelo}
\paragraph{Ventajas}
\paragraph{Desventajas}

\section{Control de Versiones}
\section{Redes Sociales}
\section{Edición Musical}
