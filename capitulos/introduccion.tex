\section{Antecedentes}

La música, como cualquier otro campo, se ha beneficiado de increíbles
contribuciones musicales tanto en el plano solista como en el plano grupal.
Compositores como Beethoven o Mozart dejaron memorables sinfonías y sonatas,
mientras que grupos como los Beatles dejaron grandes baladas. Dar fruto a estas
increíbles obras fueron fruto del esfuerzo y desde luego, de la colaboración
directa e indirecta de varias personas. Sin la influencia de otros artistas un
músico no tiene un punto de partida firme. El ascendiente de varias personas
hacen más enriquecedora la creación,  más elevado su resultado.  La inspiración
puede venir de diversas fuentes, muchas de las cuales no son siempre las
habituales. Sin la inspiración, la creación no fluye. Se dice que el 99% del
trabajo de un artista es buscarla constantemente, y el 1% es el trabajo fruto
de haberla encontrado.

En cualquier tipo de trabajo en grupo, al sumar esfuerzos la creatividad se
multiplica. Pero no siempre podemos estar de acuerdo con toda la lluvia de
ideas de un grupo heterogéneos de personas. Así, en el plano musical el fruto
de todo el trabajo se logra en base al consenso. Se desechan unas propuestas y
se mejoran otras con el fin de obtener el mejor producto final, en este caso
una canción.

A medida que avanza la tecnología, y las formas en las que interactuamos como
individuos, se multiplican las ideas. Hemos visto como del fruto de una simple
plataforma web para compartir videos, se han creado nuevas formas de expresión
en el campo audiovisual. Así por ejemplo se han generado aplicaciones para la
edición de videos con otras personas.

En el campo musical este es un tema que se está tratando de explotar. Si bien
existen aplicaciones web que buscan ser un centro de colaboración musical,
están atados a aplicaciones de edición musical de terceros, con lo que el
proceso se vuelve menos intuitivo para el usuario. De igual manera en
aplicaciones con sistemas de edición propios, no se aplica obligatoriamente un
sistema de versiones , lo cual facilitaría la colaboración entre varios
artistas.

Por los motivos mencionados se ha planteado este sistema, que servirá como
punto de colaboración, creación y edición para artistas.

\section{Alcance}

El sistema que se plantea es una red social implementada en una plataforma web,
en la cual se incentivará la colaboración musical entre artistas. Se crearán
permisos para saber que personas pueden realizar ediciones de una canción. El
creador de la canción tendrá la potestad de dar estos permisos.  Un autor podrá
aceptar o rechazar cambios hechos a la canción en cuestión, así como deshacer
cambios y volver a un punto en la edición de la canción. Para esto se usará un
sistema de control de versiones ya existente y se lo adaptará a las necesidades
de la página web.

La parte principal de la aplicación será la edición de la canción en si. Se
podrá optar entre añadir pistas que tendrán contenido de batería, piano, etc.
También se podrá cambiar la distribución de los mismos a lo largo de la canción
de una manera intuitiva para el usuario.

Los sonidos para las pistas serán suministrados por la página, el usuario
utilizará estas librerías y podrá editar las notas y tiempos para su pista.

Para el desarrollo se utilizara la metodología SCRUM. Es un campo todavía no
muy explotado que permitirá generar nuevas ideas a medida que  se desarrolle el
tema, permitiendo a su vez una mejor respuesta a los cambios que requiera la
aplicación.

Se hará uso también de la técnica de programación TDD o Test Driven
Development. Este es un proceso de desarrollo de software que se basa en la
idea de desarrollar pruebas, codificar y refactorizar el código construido.
(Barrio, 2012).

Como herramienta adicional, se usará un marco de desarrollo que sigue el patrón
de arquitectura MVC (Modelo Vista Controlador) el cual es un patrón que define
la organización independiente del Modelo (Objetos de Negocio), la Vista
(interfaz con el usuario u otro sistema) y el Controlador (controlador del
workflow de la aplicación).  (LDI Universidad Carlos III de Madrid, 2014).

\section{Justificación}

En lo que se refiere a música, actualmente existen muchas paginas para escoger,
lo cual incentiva su difusión.  En cuanto a la colaboración musical, podemos
encontrar las siguientes aplicaciones web, cada una con sus pros y sus contras:
Sistemas como Blend (https://blend.io/) ; Wavestack
(http://www.wavestack.com/). Los dos se enfocan en ser repositorios de música,
pero no realizan un manejo de control de versiones. La aplicación web Wavestack
no integra algún sistema de edición musical , como es el caso de Blend.
Existen otras páginas web como Splice (https://www.splice.com) que nos permiten
colaborar con otros artistas e incluso integran un control de versiones, pero
necesitan acoplarse a un software de edición ya existente. Esta situación puede
llegar a ser confusa y no muy amigable para el usuario final , que requerirá
primero configurar la aplicación en su equipo para empezar a usar los servicios
de la pagina.

\section{Objetivos}

\subsection{Objetivo General}

En lo que se refiere a música, actualmente existen muchas paginas para escoger,
lo cual incentiva su difusión.  En cuanto a la colaboración musical, podemos
encontrar las siguientes aplicaciones web, cada una con sus pros y sus contras:
Sistemas como Blend (https://blend.io/) ; Wavestack
(http://www.wavestack.com/). Los dos se enfocan en ser repositorios de música,
pero no realizan un manejo de control de versiones. La aplicación web Wavestack
no integra algún sistema de edición musical , como es el caso de Blend.
Existen otras páginas web como Splice (https://www.splice.com) que nos permiten
colaborar con otros artistas e incluso integran un control de versiones, pero
necesitan acoplarse a un software de edición ya existente. Esta situación puede
llegar a ser confusa y no muy amigable para el usuario final , que requerirá
primero configurar la aplicación en su equipo para empezar a usar los servicios
de la pagina.

\subsection{Objetivos Específicos}

\begin{itemize}
  \item Implementar y adaptar un sistema de edición en línea que permitirá editar la pista como tal
  \item Adaptar un sistema de control de versiones y adaptarlo conjuntamente con el sistema de edición
  \item Diseñar un sistema de permisos para grupos y sus miembros
  \item Crear pruebas unitarias y de integración que nos servirá como documentación del código creado
  \item Crear una documentación mas extensa de las historias de usuario por medio de un gestor de tareas que use metodologías ágiles
\end{itemize}

\section{Metodologías}

Se utilizaran el método exploratorio y el método experimental, a lo largo del
uso de las mismas se tratara de complementar la una con la otra (una vez que se
haya explorado una posible solución se la implementara de la manera mas
optima).

El método exploratorio nos permitirá encontrar las librerías necesarias tanto
para implementar el sistema de control de versiones como la interfaz de edición
musical.

El método experimental nos ayudara a diseñar de manera mas clara la interfaz
para cada una de las características del sistema expuestas anteriormente

